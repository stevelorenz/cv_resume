%# -*- coding:utf-8 -*-
%% start of file `template_cn_blue.tex'.
%% Copyright 2006-1008 Xavier Danaux (xdanaux@gmail.com).
%%           2024-2025 Zuo Xiang (zuoxiang1991@icloud.com, xianglinks@gmail.com)
%
% This work may be distributed and/or modified under the
% conditions of the LaTeX Project Public License version 1.3c,
% available at http://www.latex-project.org/lppl/.

\documentclass[11pt,a4paper]{moderncv}


\usepackage{fontspec,xunicode}
\usepackage[slantfont,boldfont]{xeCJK}
\usepackage{xcolor}


\setmainfont{Times New Roman}  % The default English font
\setCJKmainfont{Noto Sans CJK SC}
% \defaultfontfeatures{Mapping=tex-text}
% \XeTeXlinebreaklocale "zh"
% \XeTeXlinebreakskip = 0pt plus 1pt minus 0.1pt


%-----------------------xeCJK下设置中文字体------------------------------%
\setCJKfamilyfont{song}{SimSun}  % 宋体 song
\newcommand{\song}{\CJKfamily{song}}
\setCJKfamilyfont{fs}{FangSong_GB2312}  % 仿宋2312 fs
\newcommand{\fs}{\CJKfamily{fs}}
\setCJKfamilyfont{yh}{Microsoft YaHei}  % 微软雅黑 yh
\newcommand{\yh}{\CJKfamily{yh}}
\setCJKfamilyfont{hei}{SimHei}  % 黑体  hei
\newcommand{\hei}{\CJKfamily{hei}}
\setCJKfamilyfont{hwxh}{STXihei}  % 华文细黑  hwxh
\newcommand{\hwxh}{\CJKfamily{hwxh}}
\setCJKfamilyfont{asong}{Adobe Song Std}  % Adobe 宋体  asong
\newcommand{\asong}{\CJKfamily{asong}}
\setCJKfamilyfont{ahei}{Adobe Heiti Std}  % Adobe 黑体  ahei
\newcommand{\ahei}{\CJKfamily{ahei}}
\setCJKfamilyfont{akai}{Adobe Kaiti Std}  % Adobe 楷体  akai
\newcommand{\akai}{\CJKfamily{akai}}


%------------------------------设置字体大小------------------------%
\newcommand{\chuhao}{\fontsize{42pt}{\baselineskip}\selectfont}  % 初号
\newcommand{\xiaochuhao}{\fontsize{36pt}{\baselineskip}\selectfont}  % 小初号
\newcommand{\yihao}{\fontsize{28pt}{\baselineskip}\selectfont}  % 一号
\newcommand{\erhao}{\fontsize{21pt}{\baselineskip}\selectfont}  % 二号
\newcommand{\xiaoerhao}{\fontsize{18pt}{\baselineskip}\selectfont}  % 小二号
\newcommand{\sanhao}{\fontsize{15.75pt}{\baselineskip}\selectfont}  % 三号
\newcommand{\sihao}{\fontsize{14pt}{\baselineskip}\selectfont}  % 四号
\newcommand{\xiaosihao}{\fontsize{12pt}{\baselineskip}\selectfont}  % 小四号
\newcommand{\wuhao}{\fontsize{10.5pt}{\baselineskip}\selectfont}  % 五号
\newcommand{\subwuhao}{\fontsize{10pt}{\baselineskip}\selectfont}  % 次五号
\newcommand{\xiaowuhao}{\fontsize{9pt}{\baselineskip}\selectfont}  % 小五号
\newcommand{\liuhao}{\fontsize{7.875pt}{\baselineskip}\selectfont}  % 六号
\newcommand{\qihao}{\fontsize{5.25pt}{\baselineskip}\selectfont}  % 七号


% moderncv themes
\moderncvtheme[blue]{classic}  % optional argument are 'blue' (default), 'orange', 'red', 'green', 'grey' and 'roman' (for roman fonts, instead of sans serif fonts)


% adjust the page margins
\usepackage[scale=0.9]{geometry}
% \setlength{\hintscolumnwidth}{3cm}  % if you want to change the width of the column with the dates
% \AtBeginDocument{\setlength{\maketitlenamewidth}{6cm}}  % only for the classic theme, if you want to change the width of your name placeholder (to leave more space for your address details
\AtBeginDocument{\recomputelengths}  % required when changes are made to page layout lengths


% personal data
\familyname{张勤霞}
\firstname{}
% \address{1987/01/08}{}  % optional, remove the line if not wanted
\mobile{+86~15921610367}  % optional, remove the line if not wanted
% \email{zuoxiang1991@icloud.com}  % optional, remove the line if not wanted
% \homepage{Uni Page: https://cn.ifn.et.tu-dresden.de/chair/staff/zuo-xiang/}  % optional, remove the line if not wanted
% \social[github]{GitHub: https://github.com/stevelorenz}
% \extrainfo{%
% 微信 WeChat ID:  \\
% }

\photo[96pt]{./foto_zqx.jpg}  % '64pt' is the height the picture must be resized to and 'picture' is the name of the picture file; optional, remove the line if not wanted
% \quote{China\TeX 您的LaTeX乐园,TeX\&\LaTeX 王国}  % optional, remove the line if not wante


%----------------------------------------------------------------------------------
%            content
%----------------------------------------------------------------------------------
\begin{document}
\maketitle
\vspace*{-10mm}

\section{工作经历}

\cventry{2015-2017}{上海宝格丽酒店}{中国~上海}{前台}{}
{
具备星级酒店一线服务经验,熟练掌握客房预订、办理入住与退房手续、客户咨询等全流程服务。
凭借高效专业的工作态度与扎实的专业技能和细致的服务意识,持续保持较高的客户满意度。
}

\cventry{2011-2015}{富士康电子厂}{中国~江苏~苏州}{文职岗位}{}
{
熟练处理文件归档、数据统计、行政事务协调等工作。
凭借严谨细致的工作态度与较强的执行力,有效保障部门日常工作的顺利运转。
}


\section{教育经历}

\cventry{2008-2011}{初中}{郑州东枫外国语学校~(中国~河南~郑州)}{}{}
{
顺利完成初中学业,打下扎实的英语基础。
}


\section{技能}

\cvline{\textbf{客户服务}}{具备星级酒店一线服务经验,能够高效处理客户咨询、预订、入住与退房等流程,擅长解决客户问题并提升客户满意度。}

\cvline{\textbf{沟通协调}}{良好的语言表达与人际沟通能力,擅长跨部门协调、接待客户以及处理突发情况。}

\cvline{\textbf{办公软件}}{掌握基础数据整理与文档处理技能。}

\cvline{\textbf{英语能力}}{具备良好的英语基础。}


% \subsection{Vocational}
% \cventry{year--year}{Job title}{Employer}{City}{}{Description}  % arguments 3 to 6 are optional
% \cventry{year--year}{Job title}{Employer}{City}{}{Description}  % arguments 3 to 6 are optional
% \subsection{Miscellaneous}
% \cventry{year--year}{Job title}{Employer}{City}{}{Description line 1\newline{}Description line 2}% arguments 3 to 6 are optional

% \section{Languages}
% \cvlanguage{language 1}{Skill level}{Comment}
% \cvlanguage{language 2}{Skill level}{Comment}
% \cvlanguage{language 3}{Skill level}{Comment}

% \section{Computer skills}
% \cvcomputer{category 1}{XXX, YYY, ZZZ}{category 4}{XXX, YYY, ZZZ}
% \cvcomputer{category 2}{XXX, YYY, ZZZ}{category 5}{XXX, YYY, ZZZ}
% \cvcomputer{category 3}{XXX, YYY, ZZZ}{category 6}{XXX, YYY, ZZZ}

% \section{Interests}
% \cvline{篮球}{\small 体力与技巧}
% \cvline{hobby 2}{\small Description}
% \cvline{hobby 3}{\small Description}

% \renewcommand{\listitemsymbol}{-}  % change the symbol for lists

% \section{Extra 1}
% \cvlistitem{Item 1}
% \cvlistitem{Item 2}
% \cvlistitem[+]{Item 3}  % optional other symbol% XeLaTeX can use any Mac OS X font. See the setromanfont command below.
% Input to XeLaTeX is full Unicode, so Unicode characters can be typed directly into the source.

% The next lines tell TeXShop to typeset with xelatex, and to open and save the source with Unicode encoding.

% !TEX TS-program = xelatex
% !TEX encoding = UTF-8 Unicode

% \section{Extra 2}
% \cvlistdoubleitem[\Neutral]{Item 1}{Item 4}
% \cvlistdoubleitem[\Neutral]{Item 2}{Item 5}
% \cvlistdoubleitem[\Neutral]{Item 3}{}

%% Publications from a BibTeX file
% \nocite{*}
% \bibliographystyle{plain}
% \bibliography{publications}  % 'publications' is the name of a BibTeX file

% \begin{thebibliography}{}
% \bibitem[]{} 移动增强现实可视化综述[C]. ChinaVis 2017.
% \end{thebibliography}


\end{document}


%% end of file `template_en.tex'.

%%% Local Variables:
%%% mode: latex
%%% TeX-command-extra-options: "-shell-escape"
%%% TeX-master: t
%%% TeX-engine: xetex
%%% End:
