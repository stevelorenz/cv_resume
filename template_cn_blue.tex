%# -*- coding:utf-8 -*-
%% start of file `template_cn_blue.tex'.
%% Copyright 2006-1008 Xavier Danaux (xdanaux@gmail.com).
%%           2024-2025 Zuo Xiang (zuoxiang1991@icloud.com, xianglinks@gmail.com)
%
% This work may be distributed and/or modified under the
% conditions of the LaTeX Project Public License version 1.3c,
% available at http://www.latex-project.org/lppl/.

\documentclass[11pt,a4paper]{moderncv}


\usepackage{fontspec,xunicode}
\usepackage[slantfont,boldfont]{xeCJK}
\usepackage{xcolor}


\setmainfont{Times New Roman}  % The default English font
\setCJKmainfont[BoldFont={WenQuanYi Micro Hei/Bold}]{WenQuanYi Micro Hei}
% \defaultfontfeatures{Mapping=tex-text}
% \XeTeXlinebreaklocale "zh"
% \XeTeXlinebreakskip = 0pt plus 1pt minus 0.1pt


%-----------------------xeCJK下设置中文字体------------------------------%
\setCJKfamilyfont{song}{SimSun}  % 宋体 song
\newcommand{\song}{\CJKfamily{song}}
\setCJKfamilyfont{fs}{FangSong_GB2312}  % 仿宋2312 fs
\newcommand{\fs}{\CJKfamily{fs}}
\setCJKfamilyfont{yh}{Microsoft YaHei}  % 微软雅黑 yh
\newcommand{\yh}{\CJKfamily{yh}}
\setCJKfamilyfont{hei}{SimHei}  % 黑体  hei
\newcommand{\hei}{\CJKfamily{hei}}
\setCJKfamilyfont{hwxh}{STXihei}  % 华文细黑  hwxh
\newcommand{\hwxh}{\CJKfamily{hwxh}}
\setCJKfamilyfont{asong}{Adobe Song Std}  % Adobe 宋体  asong
\newcommand{\asong}{\CJKfamily{asong}}
\setCJKfamilyfont{ahei}{Adobe Heiti Std}  % Adobe 黑体  ahei
\newcommand{\ahei}{\CJKfamily{ahei}}
\setCJKfamilyfont{akai}{Adobe Kaiti Std}  % Adobe 楷体  akai
\newcommand{\akai}{\CJKfamily{akai}}


%------------------------------设置字体大小------------------------%
\newcommand{\chuhao}{\fontsize{42pt}{\baselineskip}\selectfont}  % 初号
\newcommand{\xiaochuhao}{\fontsize{36pt}{\baselineskip}\selectfont}  % 小初号
\newcommand{\yihao}{\fontsize{28pt}{\baselineskip}\selectfont}  % 一号
\newcommand{\erhao}{\fontsize{21pt}{\baselineskip}\selectfont}  % 二号
\newcommand{\xiaoerhao}{\fontsize{18pt}{\baselineskip}\selectfont}  % 小二号
\newcommand{\sanhao}{\fontsize{15.75pt}{\baselineskip}\selectfont}  % 三号
\newcommand{\sihao}{\fontsize{14pt}{\baselineskip}\selectfont}  % 四号
\newcommand{\xiaosihao}{\fontsize{12pt}{\baselineskip}\selectfont}  % 小四号
\newcommand{\wuhao}{\fontsize{10.5pt}{\baselineskip}\selectfont}  % 五号
\newcommand{\subwuhao}{\fontsize{10pt}{\baselineskip}\selectfont}  % 次五号
\newcommand{\xiaowuhao}{\fontsize{9pt}{\baselineskip}\selectfont}  % 小五号
\newcommand{\liuhao}{\fontsize{7.875pt}{\baselineskip}\selectfont}  % 六号
\newcommand{\qihao}{\fontsize{5.25pt}{\baselineskip}\selectfont}  % 七号


% moderncv themes
\moderncvtheme[blue]{classic}  % optional argument are 'blue' (default), 'orange', 'red', 'green', 'grey' and 'roman' (for roman fonts, instead of sans serif fonts)


% adjust the page margins
\usepackage[scale=0.9]{geometry}
% \setlength{\hintscolumnwidth}{3cm}  % if you want to change the width of the column with the dates
% \AtBeginDocument{\setlength{\maketitlenamewidth}{6cm}}  % only for the classic theme, if you want to change the width of your name placeholder (to leave more space for your address details
\AtBeginDocument{\recomputelengths}  % required when changes are made to page layout lengths


% personal data
\familyname{Dr.-Ing.~相佐}
\firstname{}
\address{1991/08/21}{}  % optional, remove the line if not wanted
\mobile{+86~15829279936}  % optional, remove the line if not wanted
\email{zuoxiang1991@icloud.com}  % optional, remove the line if not wanted
\homepage{Uni Page: https://cn.ifn.et.tu-dresden.de/chair/staff/zuo-xiang/}  % optional, remove the line if not wanted
\social[github]{GitHub: https://github.com/stevelorenz}
\extrainfo{%
  LinkedIn: https://www.linkedin.com/in/zuo-xiang-480a45b4/ \\
  微信 WeChat ID: xiangzuo0821 \\
}

\photo[64pt]{myfoto.jpg}  % '64pt' is the height the picture must be resized to and 'picture' is the name of the picture file; optional, remove the line if not wanted
% \quote{China\TeX 您的LaTeX乐园,TeX\&\LaTeX 王国}  % optional, remove the line if not wante


%----------------------------------------------------------------------------------
%            content
%----------------------------------------------------------------------------------
\begin{document}
\maketitle
\vspace*{-10mm}

\section{简介}

软件工程师,电气工程博士~(研究方向:通信工程)。
博导是~Prof.~Dr.-Ing.~Dr.~h.c.~Frank~H.~P.~Fitzek。
在博士学习期间,我的研究重点是通过网络功能虚拟化~(NFV) 和
软件定义网络~(SDN) 技术实现5G 网络软件化和智能化。除了扎实的研究
技能外,我在设计和开发虚拟网络功能和服务器端应用的高性能软件系
统方面也很有经验。我在5G 云核心网络系统开发方面也有经验,对网络
系统设计、网络协议和协议栈、电信通信系统和操作系统有深入了解。
毕业后在思科~(Cisco)中国研发中心从事企业级路由器研发工作。

\section{工作经历}

\cventry{2022.09-}{Cisco 思科系统}{中国~上海}{软件工程师}{IOS-XE}
{主要参与思科~IOS-XE~系统研发,主要负责~SD-WAN~功能的设备端特性实现,多次解决系统层面的关键问题并完成~Bug~修复。
	开发工作主要涉及~ISR1K、ISR4K~以及~Cisco~1100~终端服务网关等企业路由核心平台。}

\cventry{2020.11-2021.11}{CampusGenius GmbH~(Startup Company)}{德国~德累斯顿}{开发技术顾问(兼职)}{5G Core}
{担任 5G 云核心系统开发团队的软件开发顾问~(Development Consultant)。负责开发核心网络功能基础设施框架和指导其他开发团队成员。}

\cventry{2018.03-2022.01}{德累斯顿工业大学}{德国~德累斯顿}{博士研究生(全职带薪)}{NFV,~SDN}
{研究重点是通过 NFV 和 SDN 技术实现超低延时和低功耗的软件化~5G~网络。
我的博士论文的题目是《Ultra Low-latency, Energy-efficient and Computing-centric Software Data Plane for Network Softwarization》。
攻读博士学位期间,曾担任《网络编码原理》、《计算机网络系统》、《5G 通信网络系统》等硕士课程的助教工作,并负责指导和审查多篇硕士毕业论文。}

\cventry{2016.10-2017.04}{爱立信~Eurolab}{德国~德累斯顿}{流媒体后端系统开发实习生}{Python}
{扩展实验室内部开发和使用的~DASH~协议研究工具
1. 降低~DASH~打包器的内存使用峰值。
2. 开发高性能的~WebDAV~服务器,用于处理具有~HTTP/1.1~分块传输编码的~DASH~片段。}

\cventry{2015.10-2016.08}{德累斯顿工业大学}{德国~德累斯顿}{教学助理~(以在读硕士生的身份)}{}
{负责《计算机原理/微机原理》和《计算机网络原理》课程的实验设计和指导。
1. 用 C 语言和汇编语言为树莓派单板机上的主要外围设备 (例如温度传感器和数字显示器) 开发驱动和控制程序。
2. 用 Mininet 仿真器评估和分析不同的 TCP 拥堵算法的延迟性能。}

\section{教育经历}

\cventry{2018.03-2022.08}{工学博士~(Dr.-Ing.)}{德累斯顿工业大学~(德国~德累斯顿)}{电子信息工程专业}{}
{研究重点是通过 NFV 和 SDN 技术实现超低延时和低功耗的软件化 5G 网络。
以优异的成绩顺利通过答辩并获得博士学位。}

\cventry{2014.10-2017.12}{工学硕士~(Diplom.-Ing.)}{德累斯顿工业大学~(德国~德累斯顿)}{电子信息工程专业}{}
{以优异的成绩毕业: GPA:3.6/4.0, 被教授邀请继续攻读博士学位。}

\cventry{2018.03-2022.08}{工学学士}{西安电子科技大学~(中国~西安)}{通信工程专业}{}
{以优异的成绩毕业: GPA:3.5/4.0, 获得成绩优异奖学金两次。}


\section{项目}
\cvline{5G Cloud Core}
{为企业园区网络 5G 部署开发的云核心网络系统。我负责系统设计、开发 (Golang) 和指导开发团队 (五人)。}

\cvline{ComNetsEmu}
{一个用于 NFV/SDN/COIN 应用的开源网络仿真器。
我负责这个项目的所有的核心模块和应用样本的开发~(Python)。
该项目已于2021 年在~IEEE Communication Magazine~上发表~(请见《出版物》)。}

\cvline{X-MAN}
{一个新颖的框架实现了非侵入性和细粒度的流量工作负载监测和 CPU核心频率管理,以实现高性能软件化数据平面系统的节能。
已于 2021年在~IEEE Transactions on Network and Service Management~上发表。}

\section{技能}
\cvline{\textbf{C}}{操作系统~(Linux),网络协议栈开发~(TCP/IP, DPDK, eBPF/XDP)}
\cvline{\textbf{C++}}{流媒体,图像系统开发~(OpenCV, FFmpeg)}
\cvline{\textbf{Python}}{数据处理分析,机器学习~(Numpy, Pandas, Scikit-learn, Tensorflow)}
\cvline{\textbf{NFV/SDN}}{SDN控制器,Mininet,P4}

\section{主要学术出版物}

\cvline{2022}{Höweler, Malte, Zuo Xiang, Franz Höpfner, Giang T. Nguyen, and Frank HP Fitzek. "Towards Stateless Core Networks: Measuring State Access Patterns." (2022).}

\cvline{2021}{Wu, Huanzhuo, Jia He, Máté Tömösközi, Zuo Xiang, and Frank HP Fitzek. "In-network processing for low-latency industrial anomaly detection in softwarized networks." In 2021 IEEE Global Communications Conference (GLOBECOM), pp. 01-07. IEEE, 2021.}

\cvline{2021}{Xiang, Zuo, Malte Höweler, Dongho You, Martin Reisslein, and Frank HP Fitzek. "X-MAN: A non-intrusive power manager for energy-adaptive cloud-native network functions." IEEE Transactions on Network and Service Management 19, no. 2 (2021): 1017-1035.}

\cvline{2021}{Wu, Huanzhuo, Zuo Xiang, Giang T. Nguyen, Yunbin Shen, and Frank HP Fitzek. "Computing meets network: Coin-aware offloading for data-intensive blind source separation." IEEE Network 35, no. 5 (2021): 21-27.}

\cvline{2021}{Xiang, Zuo, Patrick Seeling, and Frank HP Fitzek. "You only look once, but compute twice: Service function chaining for low-latency object detection in softwarized networks." Applied Sciences 11, no. 5 (2021): 2177.}

\cvline{2021}{Xiang, Zuo, Sreekrishna Pandi, Juan Cabrera, Fabrizio Granelli, Patrick Seeling, and Frank HP Fitzek. "An open source testbed for virtualized communication networks." IEEE Communications Magazine 59, no. 2 (2021): 77-83.}

\cvline{2019}{Xiang, Zuo, Frank Gabriel, Elena Urbano, Giang T. Nguyen, Martin Reisslein, and Frank HP Fitzek. "Reducing latency in virtual machines: Enabling tactile Internet for human-machine co-working." IEEE Journal on Selected Areas in Communications 37, no. 5 (2019): 1098-1116.}

\cvline{2018}{Xiang, Zuo, Frank Gabriel, Giang T. Nguyen, and Frank HP Fitzek. "Latency measurement of service function chaining on OpenStack platform." In 2018 IEEE 43rd Conference on Local Computer Networks (LCN), pp. 473-476. IEEE, 2018.}
% \subsection{Vocational}
% \cventry{year--year}{Job title}{Employer}{City}{}{Description}  % arguments 3 to 6 are optional
% \cventry{year--year}{Job title}{Employer}{City}{}{Description}  % arguments 3 to 6 are optional
% \subsection{Miscellaneous}
% \cventry{year--year}{Job title}{Employer}{City}{}{Description line 1\newline{}Description line 2}% arguments 3 to 6 are optional

% \section{Languages}
% \cvlanguage{language 1}{Skill level}{Comment}
% \cvlanguage{language 2}{Skill level}{Comment}
% \cvlanguage{language 3}{Skill level}{Comment}

% \section{Computer skills}
% \cvcomputer{category 1}{XXX, YYY, ZZZ}{category 4}{XXX, YYY, ZZZ}
% \cvcomputer{category 2}{XXX, YYY, ZZZ}{category 5}{XXX, YYY, ZZZ}
% \cvcomputer{category 3}{XXX, YYY, ZZZ}{category 6}{XXX, YYY, ZZZ}

% \section{Interests}
% \cvline{篮球}{\small 体力与技巧}
% \cvline{hobby 2}{\small Description}
% \cvline{hobby 3}{\small Description}

% \renewcommand{\listitemsymbol}{-}  % change the symbol for lists

% \section{Extra 1}
% \cvlistitem{Item 1}
% \cvlistitem{Item 2}
% \cvlistitem[+]{Item 3}  % optional other symbol% XeLaTeX can use any Mac OS X font. See the setromanfont command below.
% Input to XeLaTeX is full Unicode, so Unicode characters can be typed directly into the source.

% The next lines tell TeXShop to typeset with xelatex, and to open and save the source with Unicode encoding.

% !TEX TS-program = xelatex
% !TEX encoding = UTF-8 Unicode

% \section{Extra 2}
% \cvlistdoubleitem[\Neutral]{Item 1}{Item 4}
% \cvlistdoubleitem[\Neutral]{Item 2}{Item 5}
% \cvlistdoubleitem[\Neutral]{Item 3}{}

%% Publications from a BibTeX file
% \nocite{*}
% \bibliographystyle{plain}
% \bibliography{publications}  % 'publications' is the name of a BibTeX file

% \begin{thebibliography}{}
% \bibitem[]{} 移动增强现实可视化综述[C]. ChinaVis 2017.
% \end{thebibliography}


\end{document}


%% end of file `template_en.tex'.

%%% Local Variables:
%%% mode: latex
%%% TeX-command-extra-options: "-shell-escape"
%%% TeX-master: t
%%% TeX-engine: xetex
%%% End:
